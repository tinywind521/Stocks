
\section{\fpy command line options}
\label{sec:opts}

\fpy has the following command line syntax (run \fpy without arguments
to get up to date options!!!):
\begin{verbatim}
f2py [<options>] <fortran files> [[[only:]||[skip:]] <fortran functions> ]\
                 [: <fortran files> ...]
\end{verbatim}
where 
\begin{description}
\item[\texttt{<options>}] --- the following options are available:
  \begin{description}
  \item[\texttt{-f77}]  --- \texttt{<fortran files>} are in Fortran~77
    fixed format (default).
  \item[\texttt{-f90}]  --- \texttt{<fortran files>} are in
    Fortran~90/95 free format (default for signature files).
  \item[\texttt{-fix}] --- \texttt{<fortran files>} are in
    Fortran~90/95 fixed format.
  \item[\texttt{-h <filename>}] --- after scanning the
    \texttt{<fortran files>} write the signatures of Fortran routines
    to file \texttt{<filename>} and exit. If \texttt{<filename>}
    exists, \fpy quits without overwriting the file. Use
    \texttt{-{}-overwrite-signature} to overwrite.
  \item[\texttt{-m <modulename>}] --- specify the name of the module
    when scanning Fortran~77 codes for the first time. \fpy will
    generate Python C/API module source \texttt{<modulename>module.c}.
  \item[\texttt{-{}-lower/-{}-no-lower}]  --- lower/do not lower the cases
    when scanning the \texttt{<fortran files>}. Default when
    \texttt{-h} flag is specified/unspecified (that is for Fortran~77
    codes/signature files).
  \item[\texttt{-{}-short-latex}] --- use this flag when you want to
    include the generated LaTeX document to another LaTeX document.
  \item[\texttt{-{}-debug-capi}] --- create a very verbose C/API
    code. Useful for debbuging.
%  \item[\texttt{-{}-h-force}] --- if \texttt{-h <filename>} is used then
%    overwrite the file \texttt{<filename>} (if it exists) and continue
%    with constructing the C/API module source.
  \item[\texttt{-makefile <options>}] --- run \fpy without arguments
    for more information.
  \item[\texttt{-{}-use-libs}] --- see \texttt{-makefile}.
  \item[\texttt{-{}-overwrite-makefile}] --- overwrite existing
    \texttt{Makefile-<modulename>}.
  \item[\texttt{-v}] --- print \fpy version number and exit.
  \item[\texttt{-pyinc}] --- print Python include path and exit.
  \end{description}
\item[\texttt{<fortran files>}] --- are the paths to Fortran files or
  to signature files that will be scanned for \texttt{<fortran
    functions>} in order to determine their signatures.
\item[\texttt{<fortran functions>}] --- are the names of Fortran
  routines for which Python C/API wrapper functions will be generated.
  Default is all that are found in \texttt{<fortran files>}.
\item[\texttt{only:}/\texttt{skip:}] --- are flags for filtering
  in/out the names of fortran routines to be wrapped. Run \fpy without
  arguments for more information about the usage of these flags.
\end{description}


%%% Local Variables: 
%%% mode: latex
%%% TeX-master: "f2py2e"
%%% End: 
